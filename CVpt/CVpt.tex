\documentclass[15pt,helvetica,openbib,totpages,portuguese]{europecv}
\usepackage[T1]{fontenc}
\usepackage{graphicx}
\usepackage[a4paper,top=1.27cm,left=1cm,right=1cm,bottom=2cm]{geometry}
\usepackage[portuguese,english]{babel}	
\usepackage{bibentry}
\usepackage{url}

\renewcommand{\ttdefault}{phv} % Uses Helvetica instead of fixed width font


\ecvname{\bf{Ribeiro Dias, Luis Filipe}}
\ecvfootername{Luis Filipe Dias}
\ecvaddress{Campolide, Lisboa - Portugal}	
\ecvtelephone{927856393 / 213886350}
%\ecvfax{(Remove if not relevant)}
\ecvemail{\url{luisfiliperdias@gmail.com}}
\ecvnationality{Português}
\ecvdateofbirth{30/03/1990}


%\ecvgender{(Remove if not relevant)}
%\ecvpicture[width=2cm]{mypicture}
\ecvfootnote{Mais informações em: \url{http://luisfilipedias.tk}\\  ** Este texto não foi escrito ao  abrigo do novo Acordo Ortográfico **}%\\
%\textcopyright~European Communities, 2003.}
\ecvpicture[width=3.1cm]{piccv.jpg}


\begin{document}
\selectlanguage{portuguese}

\begin{europecv}
\ecvpersonalinfo[5pt]

\ecvitem[10pt]{Carta de Condução}{Categoria B}

\ecvsection{Experiência Profissional}

\ecvitem{Datas}{Julho de 2012 – Presente}
\ecvitem{Função/Cargo Ocupado}{\textbf{Engenheiro de Hardware}}

\ecvitem{Principais actividades e responsabilidades}{\begin{itemize}
	\item Membro de uma equipa multidisciplinar, responsável pela integração de Software em equipamentos ópticos de longo alcance, tendo em vista a correcta interacção entre os vários dispositivos constituintes (SW/HW);
	\item Redação, análise e review de documentos de Especificação técnica;
	\item Execução e análise de testes de validação de Hardware (testes de Surge, EMC, testes térmicos, etc.);
	\item Interacção com equipamentos de teste, medida e caracterização de interfaces ópticas e eléctricas de alta-frequência;
	\item Compreensão e aplicação de diferentes standards/protocolos e normas internas assim como de organismos internacionais (ANSI/ETSI, ITU-T, IEEE, etc.).
	\end{itemize}
}

\ecvitem{Empregador}{\textbf{Nokia Siemens Networks/Coriant GmbH \& Co}, Lisboa - Portugal}

\ecvitem{Tipo de empresa ou sector}{Redes de dados ópticas e equipamentos de telecomunicações.}

\ecvsection{Habilitações}

\ecvitem{Datas}{Setembro de 2007 – Julho 2012}
\ecvitem{Designação da qualificação atribuída}{Mestrado Integrado em \textbf{Engenharia Electrónica e Telecomunicações}. Nota final de \textbf{15/20}.}

\ecvitem{Tese de Dissertação}{Desenvolvimento de técnicas para o aumento da eficiência de \textbf{Sistemas de Recolha de Energia Electromagnética}. Este trabalho comportou tanto o estudo dos fenómenos físicos e matemáticos associados como a componente prática de simulações e dimensionamento de sistemas reais. Foi realizado no Instituto de Telecomunicações de Aveiro. Nota final de \textbf{18/20}.}

\ecvitem{Principais disciplinas/competências profissionais}{Principais competências adquiridas no curso:
\begin{itemize}
	\item Instalação e manutenção de redes de telecomunicações;
	\item Supervisão e desenvolvimento de sistemas electrónicos de baixa e média potência, tanto para baixas frequências como para rádio-frequência e micro-ondas;
	\item Desenvolvimento e optimização de código em várias linguagens de programação;
	\item Manutenção, implementação e ajuste de sistemas de controlo;
	\item Estudo e desenvolvimento de produtos e serviços na área das telecomunicações, informática, electrónica e sistemas de controlo.
\end{itemize}
}

\ecvitem{Organização de ensino ou formação}{Universidade de Aveiro.}


\ecvitem[10pt]{\large Workshops/Concursos}{
\newline Participação na \textbf{COST} Training School: \textbf{"Energy-aware RF Circuits and Systems Design"}, na Universidade de Bolonha, em Itália, em 2012.
\newline Equipa Vencedora da competição de Engenharia local, \textbf{24h EBEC - Case Study}, em Aveiro e participação na final nacional na Universidade Nova de Lisboa, em 2012.
\newline Equipa Vice-campeã da Competição de Robótica \textbf{Micro-Rato} na edição de 2011 e participação na edição de 2012, na Universidade de Aveiro.
\newline Equipa Vencedora do \textbf{Deloitte IT Challenge} (ideias de Engenharia na área dos Sistemas de Informação), na Universidade de Aveiro em 2011 e 2012.}

\ecvsection{Aptidões e Competências Pessoais}

\ecvmothertongue[4pt]{Português}
\ecvitem{\large Outra(s) Língua(s)}{Inglês, Castelhano e Alemão \newline }
\ecvlanguageheader{(*)}
\ecvlanguage{Inglês}{Excelente}{Excelente}{Excelente}{Muito Bom}{Excelente}
\ecvlanguage{Castelhano}{Excelente}{Bom}{Bom}{Bom}{Suficiente}
\ecvlanguage{Alemão}{Suficiente}{Bom}{Suficiente}{Suficiente}{Suficiente}
%\ecvlanguage{Francês}{Suficiente}{Suficiente}{Suficiente}{Suficiente}{Suficiente}
\ecvlanguagefooter[10pt]{(*)}


\ecvitem[10pt]{\large Aptidões sociais}{Capacidade de trabalho em equipa e boa comunicação, adquiridas:
	\begin{itemize}
		\item na elaboração de diversos projectos académicos/extracurriculares e enquanto voluntário da Competição Nacional de Ciência 2012 e da CAMBADA (equipa de futebol robótico da Universidade de Aveiro).
		\item enquanto membro de uma equipa de integração de Software na Nokia Siemens Networks/Coriant GmbH \& Co;
	\end{itemize}

Fácil adaptação a novos ambientes scio-culturais com boa comunicação inter-pessoal, adquirida durante o programa de mobilidade Erasmus e como colaborador de uma empresa multinacional/multi-cultural.}


\ecvitem[10pt]{\large Aptidões e competências de organização}
{Capacidade de liderança, desenvolvida enquanto representante de equipa em vários concursos e projectos.
\newline
Boa capacidade de planeamento, cumprimento de prazos e estabelecimento de prioridades, obtidos pela entrega atempada e cuidada de todos os trabalhos propostos.
\newline
Compreensão e execução de metologias/regras próprias de uma empresa internacional de grande dimensão, enquanto colaborador da Nokia Siemens Networks.}

\ecvitem[10pt]{\large Aptidões e competências técnicas}
{Alguns dos trabalhos desenvolvidos a nível académico/hobby:
\begin{itemize}
	\item Criação de um serviço de consulta de ementas, para Restaurantes, com interface web e aplicação para Android para clientes, em 2013;
	\item Desenvolvimento várias aplicações \textbf{Android} para consulta de menus e horários, em 2012/2013;
	\item Estudo, desenvolvimento e construção de vários \textbf{Sistemas de Recolha de Energia} de ondas electromagnéticas de emissoras FM, em 2012;
	\item Construção completa de \textbf{Robot} para participação em concurso de robótica (Micro-Rato), em parceria com colega, em 2012;
	\item Desenvolvimento de \textbf{veículo telecomandado} com comunicação por módulos RF e uma unidade de processamento central (Pic32), em 2011;
	\item Projecto e Concepção de um \textbf{Regulador de Tensão} com Pic16, controlado por uma interface gráfica desenvolvida em Matlab, em 2012;
\end{itemize}
}

\ecvitem[10pt]{\large Aptidões e competências informáticas}
{
\begin{description}
	\item[Software]:
	\begin{itemize}
		\item \textit{Aplicações:} Microsoft Office, Matlab, PSpice, Microwave Office (AWR), Mentor, Advanced Design System (ADS), Proteus (ISIS \& ARES), Microwind, PICC (c/ compilador CCS), MPLAB (c/ compiladores SDCC \& Hi-Tech), HP Quality Center, Eclipse, Aptana, Netbeans, Wireshark, Design Works, Logic Works, Altium, Labview e Android SDK.
		\item \textit{Linguagens de Programação:} C, Perl, Python, TCL, Assembly, Matlab, Latex, Java, Javascript, XML, CSS3, HTML5, E(HVL).
	\end{itemize}
	\item[Hardware]:
	\begin{itemize}
		\item \textit{Equipamentos Eléctricos:} Analog/Digital/Phosphor Osciloscope, Time-Domain-Reflectometer (TDR), Communications Signal Analyser (CSA), Microwave Generator, Spectrum Analyser, Bit Error Rate Tester (BERT), Vector Network Analyser (VNA).
		\item \textit{Equipamentos Ópticos:} Optical Signal Generator/Analyser, Optical Spectrum Analyser, Optical Time-Domain-Reflectometer (OTDR), Variable Attenuator, Power Meter.
	\end{itemize}
\end{description}
}


\ecvsection{Informação Adicional}
\ecvitem[10pt]{Estudos no Estrangeiro}
{Realização de um semestre de Erasmus na Universidade Carlos III (Madrid, Espanha), no 1º semestre do 5º ano. \newline
Adquiridas competências avançadas na área da Rádio-frequência. \newline
Obtidos conhecimentos básicos em Marketing, Finanças e Contabilidade. \newline
Resultou desta experiência um aumento da capacidade de adaptação a ambientes distintos, maior autonomia e consciencialização de diferentes culturas/ tradições.}

%\bibliographystyle{plain}
%\nobibliography{publications}
%\ecvitem{}{\textbf{Publications}}
%\ecvitem{}{\bibentry{pub1}}
%\ecvitem[10pt]{}{\bibentry{pub2}}
\ecvitem{Interesses Pessoais}{Viagens, cinema, squash, ténis de mesa, culinária, arte, tecnologia, aprender novas línguas e competências.}
%\ecvitem{}{\ldots}

%\ecvsection{Annexes}
%\ecvitem{}{List any item attached to the CV}
\end{europecv}


\end{document} 